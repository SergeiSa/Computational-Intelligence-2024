

\subsection{Knowledge Areas (in terms of application)}

\begin{itemize}
    \item Robotics
    \item Automation
    \item Design (mechanical, electrical)
    \item Control
    \item Data analysis
    \item Computational geometry
\end{itemize}

\subsection{Course Delivery}

The lectures are given every week, followed by practical sessions (labs). During the sessions, students are required to develop their own code based on the knowledge acquired during lectures and the self-study.

\subsection{Prerequisite courses}

\begin{itemize}
    \item Strong prerequisites: Linear Algebra.
    \item Weak prerequisites: Calculus, Control Theory.
    \item Required background knowledge: Python (alternatively Matlab or any other language suitable to work with linear algebra-heavy problems) 
\end{itemize}

\subsection{Expected Learning outcomes}

The course will provide an opportunity for participants to:
\begin{itemize}
    \item Understand and learn to use Convex Programming (CP): Quadratic programming, Linear Programming, SOCP, SDP (LMI), as well as Mixed-integer CP.
    \item Learn how to use convex optimization solvers, especially CVX.
    \item Being able to implement optimization on practical problems, especially in robotics, control, computational geometry.
\end{itemize}

\subsection{Expected acquired core competencies}
 
\begin{itemize}
    \item Developing efficient linear algebra-based code for control and robotics applications.
    \item Developing solutions using numeric optimization, especially convex optimization.
    \item Optimization-based methods in Robotics.
\end{itemize} 

\subsection{Reference Materials}
\begin{itemize}
    \item Annotated slides
    \item Online materials
    \item Educational videos
\end{itemize}

\subsection{Computer Resources}
Students will need to run computer experiments on a laptop and/or on lab computers. 

\subsection{Laboratory Exercises} 
There are a series of labs and electronic handouts prepared for the course.

\subsection{Laboratory Resources}
Students will be required to use and modify a software tool written in Python which run on multiple platforms (Linux, Microsoft Windows, and Mac OS). The tool requires freely available software libraries.

\subsection{Cooperation Policy and Quotations}
We encourage intensive discussion and collaboration in this class. You should feel free to discuss all aspects of the class with classmates and work with them to complete your assignments and project report. However, if you are working together, you must provide details of your contribution and that of others.