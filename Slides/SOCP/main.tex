\documentclass{beamer}

\input{settings.tex}


\title{Second-order cone programming}
\subtitle{Computational Intelligence, Lecture 7}
\author{by Sergei Savin}
\centering
\date{\mydate}



\begin{document}
\maketitle


\begin{frame}{Content}

\begin{itemize}
\item  Second-order cone programming
	\begin{itemize}
		\item  SOCP to QCQP
		\item  Second-order cone constraints $\rightarrow$ to canonical form
	\end{itemize}
\end{itemize}

\end{frame}






\begin{frame}{Second-order cone programming (SOCP)}
\framesubtitle{General form}
\begin{flushleft}


The general form of a Second-order cone program (SOCP) is:

%
\begin{equation}
\begin{aligned}
& \underset{\mathbf{x}}{\text{minimize}}
& & \mathbf{f}^\top\mathbf{x}, \\
& \text{subject to}
& & \begin{cases}
    ||\mathbf{A}_i\mathbf{x} + \mathbf{b}_i||_2 \leq 
     \mathbf{c}_i^\top \mathbf{x} + d_i, \\
    \mathbf{F}\mathbf{x} = \bo{g}.
    \end{cases}
\end{aligned}
\end{equation}

LP, QP and QCQP are subsets of SOCP.
 
\end{flushleft}
\end{frame}




\begin{frame}{SOC constraints, 1}
%	\framesubtitle{General form}
	\begin{flushleft}
		
		Consider the following SOC constraint:
		
		\begin{equation}
			\label{eq:SOC}
			||\mathbf{A}\mathbf{x} + \mathbf{b}||_2 \leq 
			\mathbf{c}^\top \mathbf{x} + d
		\end{equation}
		
		Let us consider a special case when $\mathbf{x} \in \R^n$, $\mathbf{A} \in \R^{(n-1) \times n}$ and $\text{rank}\left(\begin{bmatrix}
			\mathbf{A} \\ \mathbf{c}^\top
		\end{bmatrix}\right) = n$. Then we can introduce the following substitution:
		
		\begin{equation}
			\xi = \begin{bmatrix}
				\mathbf{A} \\ \mathbf{c}^\top
			\end{bmatrix}
			\mathbf{x} + 
			\begin{bmatrix}
				\mathbf{b} \\ d
			\end{bmatrix}, 
		\ \ \ 
		\mathbf{I} = 
		\begin{bmatrix}
			\mathbf{E} \\ \mathbf{e}^\top
		\end{bmatrix}
		\end{equation}
%
where $\mathbf{I} \in\R^{n, n}$ is an identity matrix. Then constraint \eqref{eq:SOC} becomes:

		\begin{equation}
	||\mathbf{E}\xi||_2 \leq 
	\mathbf{e}^\top \xi
		\end{equation}
		
	\end{flushleft}
\end{frame}


\begin{frame}{SOC constraints, 2}
	%	\framesubtitle{General form}
	\begin{flushleft}
		
		Notice that $||\mathbf{E}\xi||_2 \leq 
		\mathbf{e}^\top \xi$ is equivalent to:
		
		\begin{equation}
			\sum\limits_{i=1}^{n-1}\xi_i^2 \leq \xi_n^2 
		\end{equation}
%	
	which is a standard form of a cone. A map back from $\xi$ to $\mathbf{x}$ is given as:
	
	\begin{equation}
		\mathbf{x} = \begin{bmatrix}
			\mathbf{A} \\ \mathbf{c}^\top
		\end{bmatrix}^{-1}
	\left(
		\xi - 
		\begin{bmatrix}
			\mathbf{b} \\ d
		\end{bmatrix}
	\right)
	\end{equation}
		
	\end{flushleft}
\end{frame}








\begin{frame}{Second-order cone programming}
\framesubtitle{Special cases}
\begin{flushleft}

We can write problem where our domain is a ball as SOCP:
%
\begin{equation}
\begin{aligned}
& \underset{\mathbf{x}}{\text{minimize}}
& & \mathbf{f}^\top\mathbf{x}, \\
& \text{subject to}
& & ||\mathbf{x}||_2 \leq d_i
\end{aligned}
\end{equation}

\bigskip

Same for ellipsoidal constraints:
%
\begin{equation}
\begin{aligned}
& \underset{\mathbf{x}}{\text{minimize}}
& & \mathbf{f}^\top\mathbf{x}, \\
& \text{subject to}
& & ||\mathbf{A}_i\mathbf{x}||_2 \leq d_i
\end{aligned}
\end{equation}
 
\end{flushleft}
\end{frame}




\begin{frame}{SOCP to QCQP}
\framesubtitle{Part 1}
\begin{flushleft}

Set $\mathbf{c}_i = 0$ and recognize that $||\mathbf{A}_i\mathbf{x} + \mathbf{b}_i||_2 \leq d_i$ is the same as $(\mathbf{A}_i\mathbf{x} + \mathbf{b}_i)^\top (\mathbf{A}_i\mathbf{x} + \mathbf{b}_i) \leq d_i^2$ (since the first implies that $d_i$ is non-negative).

\bigskip
%
\begin{equation}
\begin{aligned}
& \underset{\bo{x}}{\text{minimize}}
& & \bo{f}^\top\bo{x}, \\
& \text{subject to}
& & \begin{cases}
    \bo{x}^\top \bo{A}_i^\top \bo{A}_i \bo{x} + 
    2 \bo{b}_i^\top \bo{A}_i\bo{x} + 
    \bo{b}_i^\top \bo{b}_i  \leq d_i^2\\
    \bo{F}\bo{x} = \bo{g}.
    \end{cases}
\end{aligned}
\end{equation}

\end{flushleft}
\end{frame}




\begin{frame}{SOCP to QCQP}
\framesubtitle{Part 2}
\begin{flushleft}


Now to make the cost quadratic:
%
\begin{equation}
\begin{aligned}
& \underset{\bo{x}, t}{\text{minimize}}
& & t, \\
& \text{subject to}
& & \begin{cases}
    \bo{x}^\top \bo{A}_0^\top \bo{A}_0 \bo{x} + 
    2 \bo{b}_0^\top \bo{A}_0\bo{x} + 
    \bo{b}_0^\top \bo{b}_0  \leq t\\
    \bo{x}^\top \bo{A}_i^\top \bo{A}_i \bo{x} + 
    2 \bo{b}_i^\top \bo{A}_i\bo{x} + 
    \bo{b}_i^\top \bo{b}_i  \leq d_i^2\\
    \bo{F}\bo{x} = \bo{g}.
    \end{cases}
\end{aligned}
\end{equation}

Which is the same as:
%
\begin{equation}
\begin{aligned}
& \underset{\bo{x}}{\text{minimize}}
& & \mathbf{x}^\top \mathbf{H} \mathbf{x} + \mathbf{f}^\top\mathbf{x}, \\
& \text{subject to}
& & \begin{cases}
    \bo{x}^\top \bo{A}_i^\top \bo{A}_i \bo{x} + 
    2 \bo{b}_i^\top \bo{A}_i\bo{x} + 
    \bo{b}_i^\top \bo{b}_i  \leq d_i^2\\
    \bo{F}\bo{x} = \bo{g}.
    \end{cases}
\end{aligned}
\end{equation}

As long as $\bo{A}_0 = \sqrt{\bo{H}}$, and $\bo{b}_0 = 0.5 \bo{A}_0^{-1} \mathbf{f}$.

\end{flushleft}
\end{frame}




\begin{frame}{Friction cone}
\framesubtitle{Normal reaction force and friction}
\begin{flushleft}

\begin{figure}
    \centering
    \input{fig_twolink_2_small}
    % \caption{}
    \label{fig:contact}
\end{figure}

Let $\bo{f}$ be total reaction force, $\bo{f}_n$ be its normal component (perpendicular to the surface locally), also known as normal reaction; and let $\bo{f}_{fr}$ be its tangential component (a vector lying in the tangent plane to the surface, constructed at the contact point), or friction force. Let $\mathbf{e}_n$ be a unit vector, normal to the surface at the point of contact.

\begin{equation}
    \bo{f} =  \bo{f}_n + \bo{f}_{fr}
\end{equation}

\end{flushleft}
\end{frame}



\begin{frame}{Second-order cone programming}
\framesubtitle{Friction cone}
\begin{flushleft}

Defining $\bo{E}_t = [\mathbf{e}_{t, 1}, \ \mathbf{e}_{t, 2}] = \text{null}(\mathbf{e}_n\T)$ be an orthonormal basis in the tangential space to the surface, we can write:
%
\begin{align*}
\label{friction_cone}
    & \bo{f} = \mathbf{e}_n n + \bo{E}_t \bo{t} & \\
    & \bo{f}_n = \mathbf{e}_n n & \\
    & \bo{f}_{fr} = \bo{E}_t \bo{t} & \\
    & \bo{t} = [t_1, \ t_2]
\end{align*}
%
The friction cone conditions could be written in any of the following ways:
%
\begin{equation}
\label{friction_cone}
    \sqrt{t_1^2 + t_2^2} < \mu n
\end{equation}
%
\begin{equation}
    || \bo{E}_t^\top \mathbf{f} || \leq \mu \mathbf{e}_n^\top \mathbf{f}
\end{equation}
%
where $\mu$ is a friction coefficient.
 
\end{flushleft}
\end{frame}



\begin{frame}{Homework}
% \framesubtitle{Parameter estimation}
\begin{flushleft}

Implement a program that finds right-most point of an intersection of two ellipsoids; visualise the problem and the solution.

\end{flushleft}
\end{frame}





\myqrframe



\end{document}
