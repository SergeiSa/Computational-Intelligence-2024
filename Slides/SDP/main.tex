\documentclass{beamer}

\pdfmapfile{+sansmathaccent.map}


\mode<presentation>
{
	\usetheme{Warsaw} % or try Darmstadt, Madrid, Warsaw, Rochester, CambridgeUS, ...
	\usecolortheme{seahorse} % or try seahorse, beaver, crane, wolverine, ...
	\usefonttheme{serif}  % or try serif, structurebold, ...
	\setbeamertemplate{navigation symbols}{}
	\setbeamertemplate{caption}[numbered]
} 


%%%%%%%%%%%%%%%%%%%%%%%%%%%%
% itemize settings

\definecolor{myhotpink}{RGB}{255, 80, 200}
\definecolor{mywarmpink}{RGB}{255, 60, 160}
\definecolor{mylightpink}{RGB}{255, 80, 200}
\definecolor{mypink}{RGB}{255, 30, 80}
\definecolor{mydarkpink}{RGB}{155, 25, 60}

\definecolor{mypaleblue}{RGB}{240, 240, 255}
\definecolor{mylightblue}{RGB}{120, 150, 255}
\definecolor{myblue}{RGB}{90, 90, 255}
\definecolor{mygblue}{RGB}{70, 110, 240}
\definecolor{mydarkblue}{RGB}{0, 0, 180}
\definecolor{myblackblue}{RGB}{40, 40, 120}

\definecolor{mygreen}{RGB}{0, 200, 0}
\definecolor{mygreen2}{RGB}{245, 255, 230}

\definecolor{mygray}{gray}{0.8}
\definecolor{mydarkgray}{RGB}{80, 80, 160}

\definecolor{mydarkred}{RGB}{160, 30, 30}
\definecolor{mylightred}{RGB}{255, 150, 150}
\definecolor{myred}{RGB}{200, 110, 110}
\definecolor{myblackred}{RGB}{120, 40, 40}

\definecolor{mygreen}{RGB}{0, 200, 0}
\definecolor{mygreen2}{RGB}{205, 255, 200}

\definecolor{mydarkcolor}{RGB}{60, 25, 155}
\definecolor{mylightcolor}{RGB}{130, 180, 250}

\setbeamertemplate{itemize items}[default]

\setbeamertemplate{itemize item}{\color{myblackblue}$\blacksquare$}
\setbeamertemplate{itemize subitem}{\color{mydarkblue}$\blacktriangleright$}
\setbeamertemplate{itemize subsubitem}{\color{mygray}$\blacksquare$}

\setbeamercolor{palette quaternary}{fg=white,bg=mygblue} %mydarkgray
\setbeamercolor{titlelike}{parent=palette quaternary}

\setbeamercolor{palette quaternary2}{fg=white,bg=mygblue}%black myblue
\setbeamercolor{frametitle}{parent=palette quaternary2}

\setbeamerfont{frametitle}{size=\Large,series=\scshape}
\setbeamerfont{framesubtitle}{size=\normalsize,series=\upshape}


%%%%%%%%%%%%%%%%%%%%%%%%%%%%
% block settings

%\setbeamercolor{block title}{bg=red!50,fg=black}
%\setbeamercolor{block title}{bg=mylightblue,fg=black}
\setbeamercolor{block title}{bg=myblackblue,fg=white}

\setbeamercolor*{block title example}{bg=mygreen!40!white,fg=black}

\setbeamercolor*{block body example}{fg= black,
	bg= mygreen2}


%%%%%%%%%%%%%%%%%%%%%%%%%%%%
% URL settings
\hypersetup{
	colorlinks=false,
	linkcolor=blue,
	filecolor=blue,      
	urlcolor=blue,
}

%%%%%%%%%%%%%%%%%%%%%%%%%%

\renewcommand{\familydefault}{\rmdefault}

\usepackage{amsmath}
\usepackage{mathtools}

\usepackage{subcaption}

\usepackage{qrcode}

\DeclareMathOperator*{\argmin}{arg\,min}
\newcommand{\bo}[1] {\mathbf{#1}}

\newcommand{\dx}[1] {\dot{\mathbf{#1}}}
\newcommand{\ma}[4] {\begin{bmatrix}
		#1 & #2 \\ #3 & #4
\end{bmatrix}}
\newcommand{\myvec}[2] {\begin{bmatrix}
		#1 \\ #2
\end{bmatrix}}
\newcommand{\myvecT}[2] {\begin{bmatrix}
		#1 & #2
\end{bmatrix}}

\newcommand{\R}{\mathbb{R}} 
\newcommand{\T}{^\top}     



\newcommand{\mydate}{Spring 2023}

\newcommand{\mygit}{\textcolor{blue}{\href{https://github.com/SergeiSa/Computational-Intelligence-Slides-Spring-2023}{github.com/SergeiSa/Computational-Intelligence-Slides-Spring-2023}}}

\newcommand{\myqr}{ \textcolor{black}{\qrcode[height=1.5in]{https://github.com/SergeiSa/Computational-Intelligence-Slides-Spring-2023}}
}

\newcommand{\myqrframe}{
	\begin{frame}
		\centerline{Lecture slides are available via Github, links are on Moodle}
		\bigskip
		\centerline{You can help improve these slides at:}
		\centerline{\mygit}
		\bigskip
		\myqr
	\end{frame}
}


\newcommand{\bref}[2] {\textcolor{blue}{\href{#1}{#2}}}



%%%%%%%%%%%%%%%%%%%%%%%%%%%%
% code settings

\usepackage{listings}
\usepackage{color}
% \definecolor{mygreen}{rgb}{0,0.6,0}
% \definecolor{mygray}{rgb}{0.5,0.5,0.5}
\definecolor{mymauve}{rgb}{0.58,0,0.82}
\lstset{ 
	backgroundcolor=\color{white},   % choose the background color; you must add \usepackage{color} or \usepackage{xcolor}; should come as last argument
	basicstyle=\footnotesize,        % the size of the fonts that are used for the code
	breakatwhitespace=false,         % sets if automatic breaks should only happen at whitespace
	breaklines=true,                 % sets automatic line breaking
	captionpos=b,                    % sets the caption-position to bottom
	commentstyle=\color{mygreen},    % comment style
	deletekeywords={...},            % if you want to delete keywords from the given language
	escapeinside={\%*}{*)},          % if you want to add LaTeX within your code
	extendedchars=true,              % lets you use non-ASCII characters; for 8-bits encodings only, does not work with UTF-8
	firstnumber=0000,                % start line enumeration with line 0000
	frame=single,	                   % adds a frame around the code
	keepspaces=true,                 % keeps spaces in text, useful for keeping indentation of code (possibly needs columns=flexible)
	keywordstyle=\color{blue},       % keyword style
	language=Octave,                 % the language of the code
	morekeywords={*,...},            % if you want to add more keywords to the set
	numbers=left,                    % where to put the line-numbers; possible values are (none, left, right)
	numbersep=5pt,                   % how far the line-numbers are from the code
	numberstyle=\tiny\color{mygray}, % the style that is used for the line-numbers
	rulecolor=\color{black},         % if not set, the frame-color may be changed on line-breaks within not-black text (e.g. comments (green here))
	showspaces=false,                % show spaces everywhere adding particular underscores; it overrides 'showstringspaces'
	showstringspaces=false,          % underline spaces within strings only
	showtabs=false,                  % show tabs within strings adding particular underscores
	stepnumber=2,                    % the step between two line-numbers. If it's 1, each line will be numbered
	stringstyle=\color{mymauve},     % string literal style
	tabsize=2,	                   % sets default tabsize to 2 spaces
	title=\lstname                   % show the filename of files included with \lstinputlisting; also try caption instead of title
}

%%%%%%%%%%%%%%%%%%%%%%%%%%%%
% tikz settings

\usepackage{tikz}
\tikzset{every picture/.style={line width=0.75pt}}

%%%%%%%%%%%%%%%%%%%%%%%%%%%%




\title{Semidefinite Programming}
\subtitle{Computational Intelligence, Lecture 9}
\author{by Sergei Savin}
\centering
\date{\mydate}



\begin{document}
\maketitle


\begin{frame}{Content}

\begin{itemize}
\item  Semidefinite Programming (SDP)
\item  Schur compliment
\item  SOCP as SDP
\item  Eigenvalues
\item  Continuous Lyapunov equation as an SDP/LMI
\item  Discrete Lyapunov equation as an SDP/LMI
%\item  Example 3:  LMI Controller design for continuous LTI
\item How to describe an ellipsoid
\item Volume of an ellipsoid
\item Inscribed ellipsoid algorithms
%\item Homework
\end{itemize}
\end{frame}



\begin{frame}{Semidefinite Programming (SDP)}
%\framesubtitle{General form}
\begin{flushleft}

General form of a semidefinite program is:

%
\begin{equation}
\begin{aligned}
& \underset{\mathbf{x}}{\text{minimize}}
& & \mathbf{c}^\top\mathbf{x}, \\
& \text{subject to}
& & \begin{cases}
    \mathbf{G} + \sum \mathbf{F}_i x_i \preceq 0, \\
    \mathbf{A}\mathbf{x} = \mathbf{b}.
    \end{cases}
\end{aligned}
\end{equation}

where $\mathbf{F}_i \succeq 0$ and $\mathbf{G} \succeq 0$ (meaning they are positive semidefinite).

\bigskip

Constraint $\mathbf{G} + \sum \mathbf{F}_i x_i \preceq 0$ is called \emph{linear matrix inequality} or \emph{LMI}.
 
\end{flushleft}
\end{frame}




\begin{frame}{SDP - Multiple LMI}
%\framesubtitle{Multiple LMI}
\begin{flushleft}

SDP can have several LMIs. Assume you have:

\begin{equation}
    \begin{cases}
        \mathbf{G} + \sum \mathbf{F}_i x_i \preceq 0 \\
        \mathbf{D} + \sum \mathbf{H}_i x_i \preceq 0
    \end{cases}
\end{equation}


This is equivalent to:

\begin{equation}
    \begin{bmatrix} 
            \mathbf{G} & \mathbf{0} \\
            \mathbf{0} & \mathbf{D}
    \end{bmatrix} +
    \sum
    \begin{bmatrix} 
            \mathbf{F}_i & \mathbf{0} \\
            \mathbf{0}   & \mathbf{H}_i
    \end{bmatrix}
    x_i \preceq 0
\end{equation}

\end{flushleft}
\end{frame}




\begin{frame}{SDP decision variable}
%\framesubtitle{}
\begin{flushleft}

Sometimes it is easier to directly think of semidefinite matrices as of decision variables. This leads to programs with such formulation:
%
\begin{equation}
\begin{aligned}
& \underset{\mathbf{X}}{\text{minimize}}
& & \text{tr}(\mathbf{E}\T \mathbf{X}), \\
& \text{subject to}
& & \begin{cases}
     \text{tr}(\mathbf{A}_i\T\mathbf{X}) =  \mathbf{b}_i, \\
     \mathbf{C}\mathbf{X} \preceq \mathbf{D}.
    \end{cases}
\end{aligned}
\end{equation}

where cost and constraints should adhere to SDP limitations.

\end{flushleft}
\end{frame}



\begin{frame}{Trace of a matrix product}
	%\framesubtitle{}
	\begin{flushleft}
		
		Consider a matrices $\bo{E} = \begin{bmatrix}
			\bo{e}_1 & ... & \bo{e}_n
		\end{bmatrix}$ and 
		$\bo{X} = \begin{bmatrix}
			\bo{x}_1 & ... & \bo{x}_n
		\end{bmatrix}$. Their product can be written as:
		
		\begin{equation}
			\mathbf{E}\T \mathbf{X} = 
			\begin{bmatrix}
				\bo{e}_1 \T \bo{x}_1 & \bo{e}_1 \T \bo{x}_2 & ... \\
				\bo{e}_2 \T \bo{x}_1 & \bo{e}_2 \T \bo{x}_2 & ... \\
				... & ... & ... 
			\end{bmatrix}
		\end{equation}
		
		Thus, the trace of this product is given as:
		
		%
		\begin{equation}
			\text{tr}(\mathbf{E}\mathbf{X}) =  \bo{e}_1 \T \bo{x}_1 + ... + \bo{e}_n \T \bo{x}_n
		\end{equation}
		
		We can see that this is equivalent to an element-wise dot product. 
		
		\bigskip
		
		In a cost function, matrix $\bo{E}$ plays the role of weights, similar to $\bo{f}$ in the linear cost $\bo{f}\T \bo{x}$. Quadratic cost can be expressed as $\bo{X}\T\bo{X}$.
		
	\end{flushleft}
\end{frame}



\begin{frame}{Continuous Lyapunov eq. as SDP/LMI (1)}
	%\framesubtitle{Mathematical formulation}
	\begin{flushleft}
		
		In control theory, Lyapunov equation is a condition of whether or not a continuous LTI system $\dot{\mathbf{x}} = \mathbf{A}\mathbf{x}$ is stable:
		
		\begin{equation}
			\begin{cases}
				\mathbf{A}^\top\mathbf{P} + \mathbf{P}\mathbf{A}  \preceq -\mathbf{Q} \\
				\mathbf{P} \succeq 0
			\end{cases}
		\end{equation}
		%
		where $\mathbf{Q} \succeq 0$ is a constant and decision variable is $\mathbf{P}$. This can be represented as an SDP:
		%
		\begin{equation}
			\begin{aligned}
				& \underset{\mathbf{P}}{\text{minimize}}
				& & 0, \\
				& \text{subject to}
				& & \begin{cases}
					\mathbf{P} \succeq 0, \\
					\mathbf{A}^\top\mathbf{P} + \mathbf{P}\mathbf{A} + \mathbf{Q} \preceq 0.
				\end{cases}
			\end{aligned}
		\end{equation}
		
		
	\end{flushleft}
\end{frame}




\begin{frame}{Continuous Lyapunov eq. as SDP/LMI (2)}
	%\framesubtitle{Code}
	\begin{flushleft}
		
		\begin{lstlisting}[language=Matlab]
n = 5; V = randn(n, 2);
% Connectivity:
C = [1, 2; %edge 1
1, 3; %edge 2
2, 3; %edge 3
2, 4; %edge 4
3, 5; %edge 5
4, 5];%edge 6
nc = size(C, 1);
d = zeros(nc, 1);  %cost - distance
for i = 1:nc
	d(i) = norm(V(C(i, 2), :) - V(C(i, 1), :));
end
\end{lstlisting}

		
	\end{flushleft}
\end{frame}




\begin{frame}{Discrete Lyapunov eq. as SDP/LMI (1)}
	%\framesubtitle{Mathematical formulation}
	\begin{flushleft}
		
		In control theory, Discrete Lyapunov equation is a condition of whether or not a discrete LTI system $\mathbf{x}_{i+1} = \mathbf{A}\mathbf{x}_i$ is stabilizable:
		
		\begin{equation}
			\begin{cases}
				\mathbf{A}^\top \mathbf{P}\mathbf{A} - \mathbf{P} + \mathbf{Q} \preceq 0 \\
				\mathbf{P} \succeq 0
			\end{cases}
		\end{equation}
		%
		where $\mathbf{Q} \succeq 0$ is a constant and decision variable is $\mathbf{P}$. This can be represented as an SDP:
		%
		\begin{equation}
			\begin{aligned}
				& \underset{\mathbf{P}}{\text{minimize}}
				& & 0, \\
				& \text{subject to}
				& & \begin{cases}
					\mathbf{P} \succeq 0, \\
					\mathbf{A}^\top \mathbf{P}\mathbf{A} - \mathbf{P} + \mathbf{Q} \preceq 0.
				\end{cases}
			\end{aligned}
		\end{equation}
		
		
	\end{flushleft}
\end{frame}



\begin{frame}{Discrete Lyapunov eq. as SDP/LMI (2)}
	%\framesubtitle{Code}
	\begin{flushleft}
		
		
\begin{lstlisting}[language=Matlab]

V = randn(10, 2);
k = convhull(V);
P = V(k, :);

[domain_A, domain_b] = vert2con(P);
norm_A = vecnorm(domain_A');

f = [-1; 0; 0];
A = [reshape(norm_A, [], 1), domain_A];
b = domain_b;

x = linprog(f, A,b, [], []);

center = [x(2),x(3)];
r = x(1);
\end{lstlisting}
		
	\end{flushleft}
\end{frame}








\begin{frame}{Schur compliment, 1}
	%	\framesubtitle{Part 1}
	\begin{flushleft}
		
		Schur compliment. Given $\bo{M}$
		
		\begin{equation}
			\bo{M} = 
			\begin{bmatrix}
				\bo{A} & \bo{B} \\
				\bo{B}^\top & \bo{C}
			\end{bmatrix}
		\end{equation}
		
		with full-rank $\bo{A}$, we can make the following statement:
		
		\begin{itemize}
			\item $\bo{M} \succ 0$ iff $\bo{A} \succ 0$ and $\bo{C} - \bo{B}^\top\bo{A}^{-1}\bo{B} \succ 0$
		\end{itemize}
		
		\bigskip
		
		If $\bo{C}$ is full-rank, we can make the following statement:
		
		\begin{itemize}
			\item $\bo{M} \succ 0$ iff $\bo{C} \succ 0$ and $\bo{A} - \bo{B}\bo{C}^{-1}\bo{B}^\top \succ 0$
		\end{itemize}		
		
		
		
	\end{flushleft}
\end{frame}



\begin{frame}{Schur compliment, 2}
	%	\framesubtitle{Part 1}
	\begin{flushleft}
		
		Let us prove that $\bo{M} \succ 0$ iff $\bo{A} \succ 0$ and $\bo{C} - \bo{B}^\top\bo{A}^{-1}\bo{B} \succ 0$.
		
		\bigskip
		
		First we prove that $\bo{A} \succ 0$ and $\bo{C} - \bo{B}^\top\bo{A}^{-1}\bo{B} \succ 0$ implies $\bo{M} \succ 0$. We need to prove that the following quadratic form is positive definite:
		%
		\begin{align}
			f = 
			\begin{bmatrix}
				\bo{x}\T & 
				\bo{y}\T
			\end{bmatrix}
			\begin{bmatrix}
				\bo{A} & \bo{B} \\
				\bo{B}^\top & \bo{C}
			\end{bmatrix}
			\begin{bmatrix}
			\bo{x} \\ 
			\bo{y}
			\end{bmatrix} = \\
			= 
			\bo{x}\T \bo{A}\bo{x} + \bo{x}\T \bo{B} \bo{y} + \bo{y}\T \bo{B}\T \bo{x} + \bo{y}\T \bo{C} \bo{y}  
		\end{align}
		
		Since $\bo{C} - \bo{B}^\top\bo{A}^{-1}\bo{B} \succ 0$, the following quadratic form is positive-definite:
		%
		\begin{align}
			\bo{y}\T (\bo{C} - \bo{B}^\top\bo{A}^{-1}\bo{B} ) \bo{y} > 0
		\end{align}
		
		
		
	\end{flushleft}
\end{frame}


\begin{frame}{Schur compliment, 3}
	%	\framesubtitle{Part 1}
	\begin{flushleft}
		
		We define a change of variables $\bo{x} = - \frac{1}{2} \bo{A}^{-1}\bo{B} \bo{y}$, giving us two equations:
		
		\begin{align}
			\bo{y}\T\bo{C}\bo{y}  + 2 \bo{x}\T\bo{B} \bo{y} > 0 \\
			\bo{y}\T\bo{C}\bo{y}  + 2 \bo{y}\T\bo{B}^\top \bo{x} > 0 
		\end{align}
		
		Their sum gives us: 
		
		\begin{align}
			2 \bo{y}\T\bo{C}\bo{y}  + 2 \bo{x}\T\bo{B} \bo{y} + 2 \bo{y}\T\bo{B}^\top \bo{x} > 0 \\
			\bo{y}\T\bo{C}\bo{y}  + \bo{x}\T\bo{B} \bo{y} + \bo{y}\T\bo{B}^\top \bo{x} > 0
		\end{align}
		
		Since $\bo{A} \succ 0$ we conclude that:
		
		\begin{align}
			 \bo{x}\T\bo{A}  \bo{x} +
			\bo{y}\T\bo{C}\bo{y}  + \bo{x}\T\bo{B} \bo{y} + \bo{y}\T\bo{B}^\top \bo{x} > 0
		\end{align}
		
		This finishes first part of the proof.
		
	\end{flushleft}
\end{frame}








\begin{frame}{SOCP as SDP}
%	\framesubtitle{Part 1}
	\begin{flushleft}
		
		Let us prove that SOCP is a sub-set of SDP. SOC constraint is:
		
		\begin{equation}
			||\bo{A}\bo{x} + \bo{b}|| \leq \bo{c}^\top \bo{x} + d
		\end{equation}
		%
		where $\bo{c}^\top \bo{x} + d \geq 0$, and we can rewrite the SOC as: $(\bo{A}\bo{x} + \bo{b})^\top (\bo{A}\bo{x} + \bo{b}) = (\bo{c}^\top \bo{x} + d)^2$, and assuming $\bo{c}^\top \bo{x} + d > 0$ we can write it as:
		
		\begin{equation}
			\frac{(\bo{A}\bo{x} + \bo{b})^\top (\bo{A}\bo{x} + \bo{b})}{\bo{c}^\top \bo{x} + d} \leq \bo{c}^\top \bo{x} + d
		\end{equation}
		
		which is equivalent to:
		
		\begin{equation}
			-\frac{(\bo{A}\bo{x} + \bo{b})^\top (\bo{A}\bo{x} + \bo{b})}{-(\bo{c}^\top \bo{x} + d)} \leq \bo{c}^\top \bo{x} + d
		\end{equation}
		
	\end{flushleft}
\end{frame}





\begin{frame}{SOCP as SDP}
%	\framesubtitle{Part 2}
	\begin{flushleft}
		
		Note that $-\frac{(\bo{A}\bo{x} + \bo{b})^\top (\bo{A}\bo{x} + \bo{b})}{-(\bo{c}^\top \bo{x} + d)} \leq \bo{c}^\top \bo{x} + d$ is equivalent to:
		
		\begin{equation}
			-\frac{(\bo{A}\bo{x} + \bo{b})^\top (\bo{A}\bo{x} + \bo{b})}{(\bo{c}^\top \bo{x} + d)} + (\bo{c}^\top \bo{x} + d) \geq 0
		\end{equation}
		
		Using Schur we can re-write it as:
		
		\begin{equation}
			\begin{bmatrix}
				(\bo{c}^\top \bo{x} + d) & (\bo{A}\bo{x} + \bo{b}) \\
				(\bo{A}\bo{x} + \bo{b})^\top & (\bo{c}^\top \bo{x} + d)
			\end{bmatrix} \succeq 0
		\end{equation}
		
		which is an LMI constraint.
		
	\end{flushleft}
\end{frame}



\begin{frame}{norm and SDP}
%	\framesubtitle{Unit sphere transformation}
	\begin{flushleft}
		
		Consider the following constraint, where $\bo{X} \succeq 0$:
		
		\begin{equation}
			\label{eq:norm_LMI}
			||\bo{X}\bo{v} + \bo{b}|| \leq \bo{c}^\top \bo{x} + d
		\end{equation}
		
		Can we re-write it as an LMI? Using the same process as before we get:
		
		\begin{align}
			\begin{bmatrix}
	(\bo{c}^\top \bo{x} + d) & (\bo{X}\bo{v} + \bo{b}) \\
	(\bo{X}\bo{v} + \bo{b})^\top & (\bo{c}^\top \bo{x} + d)
		\end{bmatrix} \succeq 0
		\end{align}
	
		So, \eqref{eq:norm_LMI} is an admissible constraint in an SDP.
		
	\end{flushleft}
\end{frame}






\begin{frame}{SDP and eigenvalues}
	\begin{flushleft}
		
		Consider the problem: minimize the largest eigenvalue of $A$. The solution is:
		
		\begin{equation}
			\begin{aligned}
				& \underset{\bo{A}, t}{\text{minimize}}
				& & t, \\
				& \text{subject to}
				& & \bo{A} \preceq t\bo{I}
			\end{aligned}
		\end{equation}
		
		Proof. If $\lambda$ is an eigenvalue of $\bo{A}$, then $\bo{A} \bo{v} = \lambda\bo{v}$, hence $(\bo{A} - t\bo{I}) \bo{v} = (\lambda - t)\bo{v}$, meaning $\lambda - t$ is eigenvalue of $(\bo{A} - t\bo{I})$. Thus, if $(\bo{A} - t\bo{I})$ is negative semi-definite, then $\lambda - t \leq 0$ and $\lambda \leq t$. \qed
		
		
	\end{flushleft}
\end{frame}





\begin{frame}{How to describe an ellipsoid}
\framesubtitle{Unit sphere transformation}
\begin{flushleft}

Let us first remember how we describe a unit sphere:

\begin{equation}
    \mathcal{S} = \{ \mathbf{x}: \ || \mathbf{x} || \leq 1 \}
\end{equation}

An ellipsoid can be seen as a linear transformation of a unit sphere: 

\begin{equation}
    \mathcal{E} = \{ \mathbf{A}\mathbf{x} + \mathbf{b}: \ || \mathbf{x} || \leq 1 \}
\end{equation}
 
\end{flushleft}
\end{frame}


\begin{frame}{How to describe an ellipsoid}
\framesubtitle{A dual description}
\begin{flushleft}

Let us introduce a change of variables $\mathbf{z} = \mathbf{A}\mathbf{x} + \mathbf{b}$. Assuming $\mathbf{A}$ is invertible, we get:

\begin{equation}
    \mathbf{x} = \mathbf{A}^{-1}(\mathbf{z} - \mathbf{b})
\end{equation}

So, we can describe the exact same ellipsoid using an alternative formula: 

\begin{equation}
    \mathcal{E} = \{ \mathbf{z}: \ || \mathbf{B}\mathbf{z} + \mathbf{c} || \leq 1 \}
\end{equation}
 
 where $\mathbf{B} = \mathbf{A}^{-1}$ and $\mathbf{c} = -\mathbf{A}^{-1}\mathbf{b}$.
 
\end{flushleft}
\end{frame}




\begin{frame}{Volume of an ellipsoid (1)}
%\framesubtitle{Part 1}
\begin{flushleft}

For an ellipsoid of the form

\begin{equation}
    \mathcal{E} = \{ \mathbf{A}\mathbf{x} + \mathbf{b}: \ || \mathbf{x} || \leq 1 \}
\end{equation}

the "bigger" the $\mathbf{A}$, the bigger the ellipsoid. This concept can be made concrete by talking about the determinant of $\mathbf{A}$.

\bigskip

Thus, maximizing the volume of this ellipsoid is the same as maximizing $\text{det}(\mathbf{A})$. Or, it is the same as \emph{minimizing} the $\text{det}(\mathbf{A}^{-1})$, since $\text{det}(\mathbf{A}^{-1}) = 1 / \text{det}(\mathbf{A})$.

\bigskip

Finally, note that $\text{log} \ \text{det}(\mathbf{A})$ is a concave function and $\text{log} \ \text{det}(\mathbf{A}^{-1})$  is a convex function.

 
\end{flushleft}
\end{frame}




\begin{frame}{Volume of an ellipsoid (2)}
%\framesubtitle{Part 2}
\begin{flushleft}

For an ellipsoid of the form

\begin{equation}
    \mathcal{E} = \{ \mathbf{z}: \ || \mathbf{B}\mathbf{z} + \mathbf{c} || \leq 1 \}
\end{equation}

the "bigger" the $\mathbf{B}$, the \emph{smaller} the ellipsoid. We can make it obvious by thinking that increasing $\mathbf{B}$ leaves less room for valid $\mathbf{z}$, and it is the volume of valid $\mathbf{z}$ that makes the volume of the ellipsoid in this case.

\bigskip

This concept can be made concrete by talking about the determinant of $\mathbf{B}$. Thus, maximizing the volume of this ellipsoid is the same as \emph{minimizing} $\text{det}(\mathbf{B})$. Or, it is the same as \emph{maximizing} the $\text{det}(\mathbf{B}^{-1})$.
 
\end{flushleft}
\end{frame}


\begin{frame}{Min volume bounding ellipsoid}
	\begin{flushleft}
		
		Consider the problem: given V-polytope, defined by its vertices $\bo{v}_i$, find minimum-volume ellipsoid $\mathcal{E}$ containing the polytope. We will start with defining the ellipsoid as $\mathcal{E} = \{ \mathbf{z}: \ || \mathbf{B}\mathbf{z} + \mathbf{c} || \leq 1 \}$. The ellipsoid is smaller when $|| \mathbf{B} ||$ is bigger, and thus we can write the minimization as minimizing $\text{det} (\mathbf{B}^{-1}) $.
		
\begin{equation}
	\begin{aligned}
		& \underset{\mathbf{B}, \bo{c}}{\text{minimize}}
		& & \text{log} (\text{det} (\mathbf{B}^{-1}) ), \\
		& \text{subject to}
		& & \begin{cases}
			\mathbf{B} \succeq 0, \\
			|| \mathbf{B} \bo{v}_i + \mathbf{c} || \leq 1.
		\end{cases}
	\end{aligned}
\end{equation}

The solution gives us L\"owner-John ellipsoid.
		
	\end{flushleft}
\end{frame}



\begin{frame}{Max volume inscribed ellipsoid (1)}
	\begin{flushleft}
		
		Consider the problem: given H-polytope, defined by its half-spaces $\bo{a}_i^\top \bo{x} \leq b_i$, find maximum-volume ellipsoid $\mathcal{E}$ contained in the polytope. We will start with defining the ellipsoid as $\mathcal{E} = \{ \bo{C}\bo{x} + \bo{d}: \ || \bo{x} || \leq 1 \}$. The ellipsoid is larger when $|| \mathbf{C} ||$ is bigger, and thus we can write the minimization as minimizing $\text{det} (\mathbf{C}^{-1}) $.
		
		\bigskip
		
		Let us write down the constraint requiring that $\mathcal{E}$ lies in the polytope. We know that $\bo{a}_i^\top (\bo{C}\bo{x} + \bo{d}) \leq b_i$ holds for all $|| \bo{x} || \leq 1$. The worst-case scenario is when $\bo{x}$ aligned with $\bo{a}_i^\top \bo{C}$ and has length 1:
		%
		\begin{equation}
			\bo{x} = \frac{\bo{a}_i^\top \bo{C}}{|| \bo{a}_i^\top \bo{C} ||}
		\end{equation}
	%
		Thus the constraint becomes
		%
		\begin{equation}
			 || \bo{a}_i^\top \bo{C} || + \bo{a}_i^\top\bo{d} \leq b_i
		\end{equation}
		
	\end{flushleft}
\end{frame}



\begin{frame}{Max volume inscribed ellipsoid (2)}
	\begin{flushleft}
		
		Here is the resulting problem:
		
		\begin{equation}
			\begin{aligned}
				& \underset{\mathbf{C}, \bo{d}}{\text{minimize}}
				& & \text{log} (\text{det} (\mathbf{C}^{-1}) ), \\
				& \text{subject to}
				& & \begin{cases}
					\mathbf{C} \succeq 0, \\
					|| \bo{a}_i^\top \bo{C} || + \bo{a}_i^\top\bo{d} \leq b_i.
				\end{cases}
			\end{aligned}
		\end{equation}
		
		The solution gives us inscribed (inner) L\"owner-John ellipsoid.
		
	\end{flushleft}
\end{frame}




%\begin{frame}{Inscribed ellipsoid algorithms}
%\begin{flushleft}
%
%Continue with slides from Convex Optimization — Boyd \& Vandenberghe. Follow the link:
%
%\bigskip
%
%\centerline{\href{https://see.stanford.edu/materials/lsocoee364a/08GeometricalProbs.pdf}{8. Geometric problems}}
%
%\end{flushleft}
%\end{frame}



\begin{frame}{Homework}
% \framesubtitle{Parameter estimation}
\begin{flushleft}


Implement both examples from page 2 of the \href{http://stanford.edu/class/ee363/notes/lmi-cvx.pdf}{LMI CVX documents}.

\end{flushleft}
\end{frame}






\myqrframe







\end{document}
