\documentclass{beamer}

\input{settings.tex}


\title{Linear inequalities and polytopes}
\subtitle{Computational Intelligence, Lecture 5}
\author{by Sergei Savin}
\centering
\date{\mydate}

\begin{document}
\maketitle


\begin{frame}{Content}

\begin{itemize}
\item Convex polytopes
\item Half-spaces
\item H-representation
\item V-representation
\item G-representation (Zonotopes)
\item Linear approximation of convex regions
\end{itemize}

\end{frame}



\begin{frame}{Convex polytopes}
% \framesubtitle{Parameter estimation}
\begin{flushleft}

Before defining what a convex polytope is, let us look at examples:

\include{fig1}
 
\end{flushleft}
\end{frame}


\begin{frame}{Convex polytopes}
% \framesubtitle{Parameter estimation}
\begin{flushleft}

You can think of polytopes as geometric figures (or continuous sets of points) with linear edges, faces and higher-dimensional analogues.

\bigskip

\begin{definition}
 Convex polytopes are polytopes whose every two points can be connected with a line that would lie in the polytope. They can be bounded or unbounded.
\end{definition}
 
\end{flushleft}
\end{frame}


\begin{frame}{Half-spaces}
\framesubtitle{Definition}
\begin{flushleft}

We can define half-space as a set of all points $\mathbf{x}$, such that $\mathbf{a}^\top \mathbf{x} \leq b$. It has a very clear geometric interpretation. In the following image, the filled space is \textbf{not} in the half space.

\include{fig2}
 
\end{flushleft}
\end{frame}



\begin{frame}{Half-spaces}
\framesubtitle{Construction. Simple case}
\begin{flushleft}

Consider half-space that passes through the origin, and defined by its normal vector $\mathbf{n}$:

\include{fig3}

It is easy to see that this half-space can be defined as "all vectors $\mathbf{x}$, such that $\mathbf{n} \cdot \mathbf{x} \leq 0$", which is the same as using $\mathbf{n}$ instead of $\mathbf{a}$ in our original definition, setting $b = 0$.
 
\end{flushleft}
\end{frame}




\begin{frame}{Half-spaces}
\framesubtitle{Construction. General case}
\begin{flushleft}

In the general case there is some distance between the boundary of the half-space and the origin, let's say $d$.

\include{fig4}
%
Here the half space can be defined as "all vectors $\mathbf{x}$, such that $\mathbf{x}^\top \frac{\mathbf{n}}{|| \mathbf{n} ||}  \leq d$". This is the same as making $\mathbf{a} = \mathbf{n}$ and $b = d ||\mathbf{a}||$.
 
\end{flushleft}
\end{frame}



\begin{frame}{Half-spaces}
\framesubtitle{Combination}
\begin{flushleft}

We can define a region of space as an \emph{intersection} of half-spaces $\mathbf{a}_i^\top \mathbf{x} \leq b_i$:

\include{fig5}

Resulting region will be easily described as $\begin{bmatrix} \mathbf{a}_1^\top \\ ... \\ \mathbf{a}_k^\top \end{bmatrix} \mathbf{x} \leq \begin{bmatrix} b_1 \\ ... \\ b_k \end{bmatrix}$

 
\end{flushleft}
\end{frame}


\begin{frame}{H-representation of a polytope}
%\framesubtitle{}
\begin{flushleft}

The last result allows us to write any convex polytope as a matrix inequality:

\begin{equation}
\label{eq:ineq} 
    \mathbf{A} \mathbf{x} \leq  \mathbf{b} 
\end{equation}

And conversely, any matrix inequality \eqref{eq:ineq} represents either an empty set or a convex polytope.

\bigskip

\begin{definition}
 $\mathbf{A} \mathbf{x} \leq  \mathbf{b}$ is called \emph{H-representation} (half-space representation) of a polytope.
\end{definition}
 
\end{flushleft}
\end{frame}



\begin{frame}{H-representation in COP}
	%\framesubtitle{}
	\begin{flushleft}
		
		We can use containment in an H-polytope as a part of convex optimation problem. For example, the following QP includes such constraint:
		
		\begin{equation}
			\begin{aligned}
				& \underset{\mathbf{x}}{\text{minimize}}
				& & \mathbf{x}^\top \mathbf{H} \mathbf{x} + \mathbf{f}^\top\mathbf{x}, \\
				& \text{subject to}
				& & \mathbf{A}\mathbf{x} \leq \mathbf{b}.
			\end{aligned}
		\end{equation}
		
	\end{flushleft}
\end{frame}



\begin{frame}{V-representation}
% \framesubtitle{}
\begin{flushleft}

Convex polytopes have alternative representations, such as \emph{V-representation}. It amounts to representing polytope as a set of its vertices.

\begin{example}
$V = \begin{bmatrix} -1 & -1 & 1 & 1 \\ -1 & 1 & 1 & -1 \end{bmatrix}$ is a V-representation of a square.
\end{example}

\begin{example}
$\begin{bmatrix} 1 & 0 \\ 0 & 1 \\ -1 & 0 \\ 0 & -1 \end{bmatrix}
\begin{bmatrix} x_1 \\ x_2 \end{bmatrix} \leq
\begin{bmatrix} 1 \\ 1 \\ 1 \\ 1 \end{bmatrix}$
is an H-representation of the same square.
\end{example}
 
\end{flushleft}
\end{frame}


\begin{frame}{V-representation in COP}
	%\framesubtitle{}
	\begin{flushleft}
		
		We can use containment in an V-polytope as a part of convex optimation problem. For example, the following QP includes such constraint:
		
		\begin{equation}
			\begin{aligned}
				& \underset{\mathbf{x}}{\text{minimize}}
				& & \mathbf{x}^\top \mathbf{H} \mathbf{x} + \mathbf{f}^\top\mathbf{x}, \\
				& \text{subject to}
				& & \begin{cases}
					\mathbf{x} = \sum\limits_{i=1}^{n} \alpha_i  \mathbf{v}_i, \\
					\sum\limits_{i=1}^{n} \alpha_i = 1, \\
					\alpha_i \geq 0.
				\end{cases}
			\end{aligned}
		\end{equation}
		
		Notice that the constraint amounts to equating $\mathbf{x}$ to a convex combination of the vertices of the V-polytope.
		
	\end{flushleft}
\end{frame}



\begin{frame}{H and V-representations}
% \framesubtitle{}
\begin{flushleft}

To transfer from H-representation to V-representation, you need to solve \emph{vertex enumeration} problem, which is computationally expensive. 

\bigskip

It is also possible to construct H-representation out of V-representation.  Both algorithms are not convex.
 
\end{flushleft}
\end{frame}




\begin{frame}{Zonotopes: G-representation}
	% \framesubtitle{}
	\begin{flushleft}
		
		A zonotope $\mathcal{Z}$ is a symmetric polytope defined by its \emph{center} $\bo{c}$ and \emph{generator} $\bo{G}$:
		
		\begin{equation}
			\mathcal{Z} = \{ \bo{x}: \ \bo{x}=\bo{G}\beta+\bo{c}, \ ||\beta||_\infty \leq 1  \}
		\end{equation}
	
		The set $\{ \beta: \ ||\beta||_\infty \leq 1  \}$ is a hypercube and zonotope $\mathcal{Z}$ is a projection (shadow) of this hypercube onto a lower-dimensional space; the projection is defined by the matrix $\bo{G}$.
		
		% TODO: \usepackage{graphicx} required
		\begin{figure}
			\centering
			\includegraphics[width=0.4\linewidth]{zonotope_example}
			\caption{Zonotope (\bref{https://www.researchgate.net/publication/322671928_Methods_for_order_reduction_of_zonotopes}{Source}) }
			\label{fig:zonotopeexample}
		\end{figure}
		%https://www.researchgate.net/publication/322671928_Methods_for_order_reduction_of_zonotopes
		
		
	\end{flushleft}
\end{frame}




\begin{frame}{G-representation in COP}
	%\framesubtitle{}
	\begin{flushleft}
		
		We can use containment in an G-polytope as a part of convex optimation problem. For example, the following QP includes such constraint:
		
		\begin{equation}
			\begin{aligned}
				& \underset{\mathbf{x}}{\text{minimize}}
				& & \mathbf{x}^\top \mathbf{H} \mathbf{x} + \mathbf{f}^\top\mathbf{x}, \\
				& \text{subject to}
				& & \begin{cases}
					\mathbf{x} = \bo{G}\beta+\bo{c}, \\
					-1 \geq \beta_i \geq 1.
				\end{cases}
			\end{aligned}
		\end{equation}
		
	\end{flushleft}
\end{frame}




\begin{frame}{Linear approximation of convex regions}
% \framesubtitle{Parameter estimation}
\begin{flushleft}
Some convex regions can be easily approximated using polytopes.

\include{fig6}

Which allows to represent constraints on $\mathbf{x}$ to belong in such a region as a matrix inequality
 
\end{flushleft}
\end{frame}



\begin{frame}{Exercise}
% \framesubtitle{Parameter estimation}
\begin{flushleft}

Write H-representation of the following polytopes:

\begin{itemize}
    \item Equilateral triangle
    \item Square
    \item Parallelepiped
    \item Trapezoid
\end{itemize}

\end{flushleft}
\end{frame}


% \begin{frame}{Self-study}
% % \framesubtitle{Part 3}
% \begin{flushleft}

% \begin{itemize}
%     \item \href{https://www.youtube.com/watch?v=kcOodzDGV4c}{Convex Optimization, lecture 3, S. Boyd. Stanford. Convex functions}.
% \end{itemize}

% \end{flushleft}
% \end{frame}

\myqrframe

\end{document}
